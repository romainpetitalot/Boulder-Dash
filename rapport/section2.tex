\section{Exemple de section}

    \subsection{Comment apprendre le \LaTeX}
        L'usage du \LaTeX{} peut être déroutant au début (cf. figure~\ref{chien}), cette approche de la composition de document étant très loin de ce que l'on fait lorsque l'on utilise des logiciels comme Microcrotte Office ou Libre Office.
        Sans rentrer dans les détails, il faut voir le \LaTeX{} comme un outil permettant de séparer \emph{le fond} (ce que vous allez mettre dans votre document), de \emph{la forme} (comment le fond va apparaître dans le document final).

        \begin{figure}[!ht]
            \centering % Centre horizontalement tout ce qui est dans l'environnement ``figure''
                \includegraphics[width=0.5\textwidth]{images/tech-support-dog.jpg} % Chemin vers l'image à ajouter
            \caption{Toi en train d'apprendre le \LaTeX.} % Légende de l'image
            \label{chien} % Étiquette pour y faire référence ailleurs dans le document
        \end{figure}
        
        Quand vous écrivez du code \LaTeX, vous vous contentez en grande majorité de décrire le fond:
        je veux un paragraphe qui contient ce texte (on sépare deux paragraphe de texte par une ligne vide), je commence une nouvelle section/sous-section/sous-sous-section ici (en utilisant la commande \verb|\section{}|/\verb|\subsection{}|/\verb|\subsubsection{}|), je veux mettre en valeur \emph{cette partie de phrase} (en utilisant la commande \verb|\emph{}|)…
        La forme est gérée plus tard, par un modèle de document par défaut que vous pouvez éventuellement modifier indépendamment du fond.

        La meilleure manière d'apprendre à coder en \LaTeX{} est \emph{de bricoler du code \LaTeX{} et de voir ce qu'il se passe.}

        \subsubsection{Quelques bonnes pratiques}
            Je donne ici quelques lignes directrices qui, si elles sont respectées, vous éviteront un certain nombre de difficultés lors de l'écriture d'un document \LaTeX{}.
        
            \paragraph{Bien organiser son code source}
                Écrire du \LaTeX{}, c'est écrire du \emph{code source}.
                Comme vous l'avez appris à l'INSA, un bon code source doit être proprement mis en forme.
                Je vous recommande, lorsque vous écrivez du texte, d'écrire une phrase par ligne, cela rendra votre code bien plus lisible et simplifiera l'utilisation d'un logiciel de gestion de sources comme git.
                En ce qui concerne la lisibilité du code source, pensez à \emph{indenter} votre code afin de mieux vous y repérer.
            
             \paragraph{Compiler du \LaTeX{}}
                Étant donné que le \LaTeX{} est du code source, celui-ci doit être compilé (à l'aide de la commande \verb|pdflatex| par exemple).
                Comme pour un programme en Pascal, à la compilation il est possible que des messages d'avertissement ou d'erreur s'affichent.
                La plupart des messages d'avertissement ne posent pas de problèmes (le document que vous lisez en émet un certain nombre à la compilation), cependant les messages d'erreur sont beaucoup plus graves et doivent être résolus pour obtenir un document final correct et complet.
                Ce n'est pas par ce qu'à la fin de votre compilation vous obtenez un document que la compilation s'est passée sans erreur.
                Dans la plupart des cas, aucun document n'est produit, mais certains éditeurs de \LaTeX{} \enquote{forcent} la compilation jusqu'au bout.
                Je vous invite donc à être très attentif à comment se déroule la compilation de votre document de manière à ne pas passer à côté d'erreurs de compilation.
                Je vous invite aussi à \emph{compiler votre document aussi souvent que possible}, cela afin de repérer le plus tôt possible les erreurs de compilation dans le code source que vous tapez.
        
        \subsubsection{Ressources externes}\label{sssect:ressources_externes}
            Pour vous aider dans votre aventure d'écriture de beaux documents, voici quelques ressources sur lesquelles vous pouvez vous appuyer pour rechercher de l'information.
            Je vous recommande d'y avoir recours quand le besoin s'en fait ressentir lorsque vous rédigez votre document.
            
            \paragraph{Wikilivres}
                Les wikilivres sur le \LaTeX{} (en français et en anglais) sont particulièrement bien rédigés et traitent un grand nombre de cas de figure face auxquels on peu se retrouver:
                \begin{itemize}
                \item En français: \url{https://fr.wikibooks.org/wiki/LaTeX}
                \item En anglais: \url{https://en.wikibooks.org/wiki/LaTeX}
                \item Le vadémécum du livre français vaut le détour: \url{https://fr.wikibooks.org/wiki/LaTeX/Vad%C3%A9m%C3%A9cum}
                \end{itemize}
            
            \paragraph{\LaTeX{} stack exchange}
                Le site internet \url{https://tex.stackexchange.com/} est un espace dédié aux questions/réponses sur la thématique du \LaTeX{}.
                Les réponses y sont souvent correctes et bien documentées (mais pas toujours! Faites preuve de recul.).
                Les sujets abordés varient grandement en termes de complexité, à vous de vous y retrouver.
                
                Petite mise en garde en passant: vous copiez-collez du code depuis internet \emph{à vos risques et périls}.
                Si vous ne comprenez les commandes que vous utilisez dans votre code, vous ne comprendrez pas les erreurs que celui-ci produit.
                Ce n'est pas par ce qu'une personne, sur internet, à une moment donné, à proposé une idée pour faire une chose que cette idée est la bonne pour faire ce que vous voulez.
            
            \paragraph{Le Comprehensive \TeX{} Archive Network}
                Pour les braves.
                Si vous souhaitez comprendre dans le détail comment un paquet \LaTeX{} fonctionne et comment l'utiliser, vous trouverez sa documentation sur le site \url{https://www.ctan.org/}.
                C'est souvent très instructif mais il faut être un minimum à l'aise avec le \LaTeX{} pour se plonger dans ce type de documentation.




    \subsection{Comment ne pas apprendre le \LaTeX}
        Comme dit précédemment, l'apprentissage du \LaTeX{} se fait par la \emph{pratique}.
        Essayer d'apprendre à utiliser ce langage en parcourant uniquement les références données dans la section~\ref{sssect:ressources_externes} sans jamais s'aventurer à écrire du code ne vous apportera rien.
        
        Si à un moment vous êtes bloquez et n'arrivez pas à vous en sortir même avec l'aide de vos camarades, vous pouvez solliciter de l'aide soit auprès d'un enseignant en TP, soit éventuellement sur internet.
        Soyez conscient que le code que vous avez écrit \emph{vous appartient}.
        Vous en êtes responsables.
        Si vous demandez un appui extérieur, assurez-vous de montrer que vous avez cherché à comprendre et résoudre le problème par vous-même.
        Envoyer par mail ou sur un forum un code source qui ne compile pas en demandant à ce que votre interlocuteur le répare à votre place est le meilleur moyen de vous le mettre à dos et de recevoir un lapin pancake comme réponse (cf. figure~\ref{lapin}).
        
        
        \begin{figure}[!ht]
            \centering % Centre horizontalement tout ce qui est dans l'environnement ``figure''
                \includegraphics[width=0.5\textwidth]{images/pancake-bunny.jpg} % Chemin vers l'image à ajouter
            \caption{Ma réponse quand on me pose une question sur du \LaTeX{} sans avoir essayé de chercher une solution par soi-même.} % Légende de l'image
            \label{lapin} % Étiquette pour y faire référence ailleurs dans le document
        \end{figure}
