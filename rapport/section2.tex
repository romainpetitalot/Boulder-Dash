\newpage
\section{Apport du Projet}

    \subsection{Apprentissage du \LaTeX}
        Afin de réaliser notre rapport, nous avons suivi les directives données par les professeurs, à savoir utiliser le \LaTeX{}.
        
        \subsubsection{Découverte de ce langage}
            \paragraph{Site Overleaf} Nous avons fait le choix pour ce projet d'utiliser le site/application Overleaf afin de pouvoir modifier notre rapport simultanément et travailler ensemble. De plus, Overleaf permet de compiler le code directement afin d'avoir un aperçu sur la moitié de l'écran, ce qui est extrêmement pratique lorsque l'on commence à coder en \LaTeX{} car il n'est pas aisé de se représenter le rendu. Enfin, ce site fonctionne par projet, il est donc facile de partager tout le projet à son groupe et de travailler avec les mêmes fichiers indispensables tels que des images.
            
            \paragraph{Avantage du \LaTeX{}} Le \LaTeX{} a été quasiment indispensable pour la rédaction de notre cahier des charges et du rapport. Effectivement, nous avons dû insérer du code, qu'il s'agisse de pseudo-code ou de pascal, ce qui aurait été bien plus complexe avec un éditeur comme Word ou Libre office. Le \LaTeX{} permet également des rendus de qualité et il est important pour nous de découvrir ce langage.
            
            \paragraph{Les bases du \LaTeX{}} Pour la plupart des membres du groupe, nous n'avions que très peu ou jamais codé en \LaTeX{}. Ainsi, la découverte réelle de ce langage a été réalisée lors de l'élaboration du cahier des charges. Ce dernier étant assez rapide et condensé, il a permis de découvrir doucement les bases du \LaTeX{}. 

        
        \subsubsection{Approfondissement}
            \paragraph{Besoins supplémentaires} Après avoir acquis les bases indispensables à la création d'un document telles que l'ajout de titres, sections, sous-sections, paragraphes ; nous avons dû apprendre à insérer des images, des algorithmes, des parties de code. L'essentiel de cet apprentissage s'est fait grâce aux explications que l'on peut trouver sur le net. En outre, rédiger un rapport sur un seul document \LaTeX{} n'aurait pas été lisible et compréhensible, d'autant plus que nous l'avons rédigé à cinq. Organiser la rédaction du rapport s'est alors révélé très utile. Nous avons décomposé le rapport en plusieurs sections et avons fait les imports de packages dans un autre document. Cela a permis d'apporter beaucoup de lisibilité et de ne pas modifier involontairement les parties du code déjà réalisées.
            
            
            \paragraph{Difficultés rencontrées} Nous avons pu utiliser un modèle de rapport en \LaTeX{} fourni par les professeurs. Celui-ci nous a beaucoup aidé notamment pour la structure du rapport. Cependant, nous avons tout de même rencontré des difficultés comme pour importer des images, notamment l'analyse descendante qui a une taille importante, mais aussi l'insertion de code. Pour réussir à surpasser la difficulté de ce dernier point, certains membres du groupe ont appris à utiliser le package minted. 
            \newline
            De plus, nous avons rencontré de nombreuses difficultés lors de l'implémentation de notre jeu, notamment, avec l'utilisation de la SDL. En effet, dans un soucis de réalisme, pour que notre jeu soit le plus agréable possible à jouer, la gestion de l'affichage et des délais des objets de notre jeu était la partie la plus compliquée à implémenter, particulièrement lors des éboulements. De plus, il y avait beaucoup de petits détails à prendre en compte qui ne paraissaient pas très importants individuellement, mais qui, additionnés, transforment l'expérience de jeu, par exemple, le curseur quand on appuie sur échap en pleine partie, ou la gestion du chrono et des pauses quand on est dans le menu. 
                



    \subsection{Travail en groupe}
        
        \subsubsection{Découverte et utilisation de Gitlab} L'utilisation de Gitlab nous a été imposé par nos professeurs. Nous n'avions jamais utilisé Gitlab et les débuts sur ce logiciel étaient un peu compliqués, notamment pour comprendre le fonctionnement des différentes commandes. Malgré ces quelques complications, nous avons pu apprendre, à force d'utiliser les commandes, à comprendre leurs fonctions. 
        
        
        \subsubsection{Mise en accord sur les fonctionnalités} Pour les différentes fonctionnalités attendues, chacun a proposé ses idées et nous nous sommes mis d'accord pour savoir si c'était réalisable et si cela rentrait dans les fonctionnalités obligatoires ou facultatives. 
        
        
        \subsubsection{Répartition du travail en fonction des aptitudes}
        Concernant la répartition du travail au sein du groupe, nous nous sommes réparti les différentes tâches en fonction des compétences de chaque membre du groupe. Les missions principales de ce projet ont été la rédaction du cahier des charges, de l'analyse descendante et du rapport, mais aussi l'écriture du code dans le langage de programmation demandé.
        
         \subsubsection{Limitations et perspectives}
        Tout d'abord, l'aspect à distance de notre projet a rendu très difficile le travail de groupe et nous avons dû nous organiser en ligne sur Discord et Messenger. Ensuite, le suivi des séances avec le professeur était tout le temps à distance et nous n'avons pas eu de véritable retour sur notre avancée sur le projet. Enfin, concernant les perspectives, nous avions établi plusieurs fonctionnalités facultatives dans notre cahier des charges, mais que, faute de temps, nous n'avons pas mis dans notre rapport. C'est pourquoi nous voudrions les reprendre par la suite pour les intégrer dans notre Boulder Dash.
        
        
